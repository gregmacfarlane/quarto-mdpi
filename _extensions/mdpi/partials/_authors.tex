$-- You can use as many custom partials as you need. Convention is to prefix name with '_'
$-- It can be useful to use such template to split some template parts in smaller pieces, which is easier to reuse. 
$-- This '_custom.tex' is used on 'title.tex' as example.
$-- See other existing format in quarto-journals/ organisation.
$-- %%%% TODO %%%%%
$-- Use it if you need to insert content at this specific place of the main Pandoc's template. Otherwise, remove it.
$-- Here we are using it to format the authors part of the template.
$-- %%%%%%%%%%%%%%%

% this solution came from ihrke/mdpi
$for(by-author/pairs)$
$if(it.value.orcid)$
\newcommand{\orcidauthor$it.key/alpha/uppercase$}{$it.value.orcid$} 
$endif$
$endfor$

\Author{$if(by-author/allbutlast)$
$for(by-author/allbutlast/pairs)$
$it.value.name.literal$$if(it.value.orcid)$\orcid$it.key/alpha/uppercase${}$endif$$sep$, 
$endfor$
~and~$for(by-author/last)$$it.name.literal$$if(it.orcid)$\orcid${ author/length/alpha/uppercase }{}$endif$$endfor$
$else$
${ by-author.name.literal } $if(by-author.orcid)$$$^{$for(by-author.affiliations)$$it.id$$sep$,$endfor$}$$\orcidA{}$endif$
$endif$}

\AuthorNames{$for(by-author)$ $by-author.name.literal$$sep$, $endfor$}
\address{myaffiliation}

\newcommand\getfirst[1]{\getfirstaux#1\relax} \def\getfirstaux#1#2\relax{#1}
\AuthorCitation{$for(by-author)$$by-author.name.family$, \getfirst{$by-author.name.given$}.$sep$; $endfor$}
% Contact information of the corresponding author
$for(by-author)$
$if(it.attributes.corresponding)$
\corres{Correspondence: $it.email$}
$endif$
$endfor$